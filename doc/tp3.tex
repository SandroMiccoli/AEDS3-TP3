
%	Documentação do Trabalho Prático 3 de AEDSIII
%	@Sandro Miccoli
%
%	* Você pode identificar erros de grafia através do seguinte comando linux:
%		aspell --encoding="utf-8" -c -t=tex --lang="pt_BR" tp2.tex
%

\documentclass[12pt]{article}
\usepackage{sbc-template}
\usepackage{graphicx}
\usepackage{latexsym}
\usepackage{subfigure}
\usepackage{times,amsmath,epsfig}
\usepackage{graphicx,url}
 \makeatletter
 \newif\if@restonecol
 \makeatother
 \let\algorithm\relax
 \let\endalgorithm\relax
\graphicspath{{./data/}}
\usepackage[lined,algonl,ruled]{algorithm2e}
\usepackage{multirow}
\usepackage[brazil]{babel}
\usepackage[utf8]{inputenc}
\usepackage{listings}
\usepackage{subfigure}

\usepackage{alltt}
\renewcommand{\ttdefault}{txtt}

\sloppy

\title{TRABALHO PRÁTICO 3: \\ Memória Virtual}

\author{Sandro Miccoli - 2009052409 - smiccoli@dcc.ufmg.br}

\address{Departamento de Ciência da Computação -- Universidade Federal de Minas Gerais (UFMG)\\
\\
\today}


\begin{document}

\maketitle

\begin{resumo}
Este relatório descreve como foi implementado uma versão simplificada de um simulador de memória virtual (SMV). Será descrito como foi modelado o problema, e como cada política de reposição de páginas funciona. Finalmente será detalhado a análise de complexidade dos algoritmos e uma análise de cada política de acordo com a especificação do trabalho, e, por último, uma breve conclusão do trabalho implementado.
\end{resumo}

\section{INTRODUÇÃO}

    A memória virtual foi inicialmente criada para possibilitar a um programa ser executado em um computador com uma quantidade de memória principal (física) menor que o tamanho de todo o espaço do utilizado pelo próprio programa. Ou seja, o espaço ocupado pelas instruções, dados e pilha de execução de um programa pode ser maior que o espaço em memória principal disponível.

    Memória virtual, é uma técnica que usa a memória secundária como uma cache para armazenamento secundário. Houve duas motivações principais: permitir o compartilhamento seguro e eficiente da memória entre vários programas e remover os transtornos de programação de uma quantidade pequena e limitada na memória principal. \cite{wikimv}

    O objetivo principal do trabalho é implementar uma versão simplificada de um sistema de memória virtual (SMV), pois usaremos apenas um nível de paginação, sem caches ou otimizações. Quando houver a necessidade de reposição de páginas, foram utilizadas três políticas de reposição:

    \begin{itemize}
    \item \textbf{FIFO (FirstIn, FirstOut)} - A página que está residente a mais tempo é escolhida para remoção.
    \item \textbf{LRU (Least Recently Used)} - A página acessada a mais tempo deve ser escolhida para remoção.
    \item \textbf{LFU (Least Frequently Used)} - A página com a menor quantidade de acessos deve ser escolhida para remoção.
    \end{itemize}


	O restante deste relatório é organizado da seguinte forma. A Seção~\ref{modelagem} descreve como foi feita a modelagem do problema e o armazenamento das páginas virtuais. A Seção \ref{solucao_proposta} descreve como foi feito a manipulação das páginas da memória e detalhes das políticas de reposição. A Seção~\ref{implementacao} trata de detalhes específicos da implementação do trabalho: quais os arquivos utilizados; como é feita a compilação e execução; além de detalhar o formato dos arquivos de entrada e saída. A Seção~\ref{avaliacao_experimental} contém a análise de desempenho de cada uma das políticas de reposição. A Seção~\ref{conclusao} conclui o trabalho.


\section{MODELAGEM}
\label{modelagem}

Para simular a memória virtual neste trabalho, foi utilizado um tipo abstrato de dados de lista duplamente encadeada. Foi escolhida esta estrutura pela facilidade de manipulação e pesquisa dos elementos contidos nela.

A abstração da estrutura é a seguinte: a lista representa a memória física e cada uma das suas células representam as páginas que foram carregadas na memória. Assim, utilizando as funções de inserção e remoção da lista encadeada, podemos implementar as políticas de reposição propostas pro trabalho.

\section{SOLUÇÃO PROPOSTA}
\label{solucao_proposta}

É fornecido na entrada do trabalho o tamanho em bytes da memória física, o tamanho de cada página e $N$ acessos à memória. Cada um desses acesso é referente à posição da memória virtual acessada sequencialmente (para mais detalhes sobre o formato de entrada, consultar o item \ref{entrada} deste trabalho). Porém, para saber em qual página cada acesso se enconta, dividimos a posição de acesso pela quantidade de bytes de cada página. Por exemplo, dada a sequência de acesso \textbf{0 2 4 2 10 1 6 8}, podemos simplificar isso para \textbf{0 0 1 0 2 0 1 2}. Assim sabemos qual página deverá estar na memória a cada acesso.

Na situação descrita acima, chegamos às páginas \textbf{0 0 1 0 2 0 1 2}; porém, com apenas 8 bytes na memória principal, podemos carregar simultâneamente apenas duas páginas, pois cada uma possui 4 bytes de tamanho. Por esse motivo precisamos adotar políticas que removam uma certa página e a troquem por outra da memória principal. Para solucionar esse problema, foi implementado um algoritmo para cada política de reposição (FIFO, LRU e LFU).

Para o \textbf{FIFO}, a escolha da página a ser removida é a que entrou primeiro, ou seja, a primeira da fila. No \textbf{LRU}, também removemos as páginas que estão no início da fila, pois toda vez que a página é acessada ela vai pro final da fila, assim, seu início contém as páginas que foram acessadas menos recentemente. No \textbf{LFU}, para cada acesso à memória, cada página possui um contador para indicar a quantidade de acessos à ela. Assim, a cada acesso à memória, ordenamos as páginas de acordo com esse contador, então as primeiras páginas da fila serão as que foram menos frequentemente usadas.

\subsection{Algoritmos implementados}

\vspace{0.2 true cm}

\begin{itemize}
 \item \begin{large}\textit{void FIFO(TipoLista * memoria, TipoCelula pagina)}\end{large}\\
 \subitem \textbf{Descrição:} "First In, First Out"; a página que entrou primeiro na memória é a primeira a ser removida.
 \subitem \textbf{Parâmetros:} Memória principal e página atual que deveria estar na memória
 \subitem \textbf{Complexidade:} $O(n)$, onde $n$ é o número de páginas que cabe na memória.
\end{itemize}

\vspace{0.2 true cm}

\begin{itemize}
 \item \begin{large}\textit{void LRU(TipoLista * memoria, TipoCelula pagina)}\end{large}\\
 \subitem \textbf{Descrição:} "Least Recently Used"; a página que foi acessada menos recentemente que será a escolha pra ser removida.
 \subitem \textbf{Parâmetros:} Memória principal e página atual que deveria estar na memória
 \subitem \textbf{Complexidade:} $O(n)$, onde $n$ é o número de páginas que cabe na memória.
\end{itemize}

\vspace{0.2 true cm}

\begin{itemize}
 \item \begin{large}\textit{void LFU(TipoLista * memoria, TipoCelula pagina)}\end{large}\\
 \subitem \textbf{Descrição:} "Least Frequently Used"; a página que é utilizada menos frequentemente que será removida.
 \subitem \textbf{Parâmetros:} Memória principal e página atual que deveria estar na memória
 \subitem \textbf{Complexidade:} $O(n)$, onde $n$ é o número de páginas que cabe na memória.
\end{itemize}

\vspace{0.2 true cm}

\begin{itemize}
 \item \begin{large}\textit{void removeMenosAcessada(TipoLista * memoria)}\end{large}\\
 \subitem \textbf{Descrição:} Procura pela página com menor número de acessos e a remove da memória.
 \subitem \textbf{Parâmetros:} Memória princial
 \subitem \textbf{Complexidade:} $O(n)$, onde $n$ é o número de páginas que cabe na memória.
\end{itemize}

\vspace{0.2 true cm}

\begin{itemize}
 \item \begin{large}\textit{TipoApontador resideEmMemoria(TipoLista * memoria, int pagina)}\end{large}\\
 \subitem \textbf{Descrição:} Percorre toda a memória para verificar se a página se encontra na memória. Retorna um apontador pra essa página.
 \subitem \textbf{Parâmetros:} Memória principal e um inteiro que representa a página a ser encontrada
 \subitem \textbf{Complexidade:} $O(n)$, onde $n$ é o número de páginas que cabe na memória.
\end{itemize}


\section{IMPLEMENTAÇÃO}
\label{implementacao}

\subsection{Código}

\subsubsection{Arquivos .c}

\begin{itemize}
\item \textbf{tp3.c} Arquivo principal do programa. Lê os arquivos de entrada, calcula os \textit{page faults} pra cada política e escreve o resultado em um arquivo de saída.
\item \textbf{lista.c} TAD da uma lista duplamente encadeada. Contém funções de manipulação (criação, inserção, remoção e liberação de memória), e também de impressão e cópia.
\item \textbf{smv.c} Contém a implementação das políticas de reposição FIFO, LRU, LFU.
\end{itemize}

\subsubsection{Arquivos .h}

\begin{itemize}
\item \textbf{lista.h} TAD da uma lista duplamente encadeada. Contém a definição da estrutura e das funções.
\item \textbf{smv.h} Contém a definição das políticas de reposição FIFO, LRU, LFU.
\end{itemize}

\subsection{Compilação}

O programa deve ser compilado através do compilador GCC através dos seguintes comandos

Para programação dinâmica:
\begin{footnotesize}
\begin{verbatim}
gcc -Wall -Lsrc src/tp3.c src/lista.c src/arquivos.c src/smv.c -o tp3 \end{verbatim}
\end{footnotesize}

Ou através do comando $make$.

\subsection{Execução}

A execução do programa tem como parâmetros:
\begin{itemize}
\item Um arquivo de entrada contendo várias instâncias a serem simuladas.
\item Um arquivo de saída que irá receber a quantidade de \textit{page faults} pra cada política de reposição
\end{itemize}

O comando para a execução do programa é da forma:

\begin{footnotesize}
\begin{verbatim} ./tp3 <arquivo_de_entrada> <arquivo_de_saída>\end{verbatim}
\end{footnotesize}


\subsubsection{Formato da entrada}
\label{entrada}

A primeira linha do arquivo de entrada contém o valor \textit{k} de instâncias que o arquivo contém. As $k$ instâncias são definidas em duas linhas cada uma. A primeira linha contem três inteiros: o tamanho em bytes da memória física, o tamanho em bytes de cada página, e o número N de acessos. A linha seguinte contém N inteiros representando as N posições da memória virtual acessadas sequencialmente.

A seguir um arquivo de entrada de exemplo:

\begin{verbatim}
1
8 4 10
0 2 4 2 10 1 0 0 6 8
\end{verbatim}

\subsubsection{Formato da saída}
\label{saida}

O arquivo de saída consiste em $k$ linhas, cada uma representando o resultado de uma instância. Cada linha contém um inteiro que representa o número de falhas utilizando FIFO, LRU e LFU, necessariamente nessa ordem. Um exemplo é mostrado abaixo:

\begin{verbatim}
6 5 5
\end{verbatim}


\section{AVALIAÇÃO EXPERIMENTAL}
\label{avaliacao_experimental}

As análises feitas para para este trabalho foram todas detalhadas na especificação. O que foi pedido foi o seguinte:

\begin{itemize}
\item Calcular a localidade de referência temporal.
\item Calcular a localidade de referência espacial.
\item Gerar o histograma das distâncias de acessos.
\item Gerar o histograma das distâncias de pilha.
\item Gerar um gráfico "Tamanho da página" x "Bytes movimentados".
\item Gerar um gráfico "Tamanho da memória" x "Falhas".
\end{itemize}

Para realizar essas análise foi diponibilizado um arquivo com uma configuração diferente no fórum da disciplina. Esse arquivo possuia apenas acessos à memória, sem informações de tamanho da memória ou da página. Foi criado dois scripts que auxiliaram na execução destas análises. O primeiro de localidade espacial e outro de localidade temporal.

Nas próximas seções iremos descrever a máquina utilizada para os testes e o resultado de cada item descrito acima.

\subsection{Máquina utilizada}
\label{maquina}

Segue especificação da máquina utilizada para os testes:
\begin{verbatim}
model name:     Intel(R) Core(TM) i3 CPU       M 330  @ 2.13GHz
cpu MHz:        933.000
cache size:     3072 KB
MemTotal:       3980124 kB
\end{verbatim}


\subsection{Localidade de referência temporal}
\label{loc_ref_temp}

A Tabela \ref{tab_ref_temp} mostra os resultados que obtivemos para o cálculo da localidade de referência temporal.

\begin{table}[h!]
\centering
\begin{footnotesize}
\begin{tabular}{|c|c|}
\hline
\textbf{Instância}  & \textbf{Temporal}  \\ \hline
1 & 19.67 \\ \hline
2 & 14.10 \\ \hline
3 & 21.08 \\ \hline
4 & 10.67 \\ \hline

\end{tabular}
\end{footnotesize}
\caption{Localidade de referência temporal \label{tab_ref_temp}}
\end{table}


\subsection{Localidade de referência espacial}
\label{loc_ref_esp}

A Tabela \ref{tab_ref_esp} mostra os resultados que obtivemos para o cálculo da localidade de referência espacial. É gritante a diferença entre o resultado da instância 2 pras outras, mas isso ocorreu pois ela tem um padrão bem peculiar. A grande maioria dos acessos são sequenciais na memória (posição 8, depois 7, depois 6, depois 5), isso caracteriza uma boa localidade de referência espacial.


\begin{table}[h!]
\centering
\begin{footnotesize}
\begin{tabular}{|c|c|}
\hline
\textbf{Instância}  & \textbf{Espacial}  \\ \hline
1 &  16.69\\ \hline
2 & 2.55\\ \hline
3 &  10.46\\ \hline
4 &  19.71\\ \hline

\end{tabular}
\end{footnotesize}
\caption{Localidade de referência temporal \label{tab_ref_esp}}
\end{table}


\subsection{Histogramas de Distância de Acessos}
\label{hist_dist_acess}


Na Figura \ref{img_dist_acess} agrupamos os quatro histogramas que foram pedidos. Logo abaixo é possível ver o resultado para cada uma das instâncias.

Na instância 1 e 4 ocorre uma distribuição maior dos acessos. Na primeira os ápices se encontram nas menores distâcias, já na outra o ápice ocorre na distância 32. As distâncais de acesso da instância 1 se encontram muito melhor distribuídas que a instância 4.

Já as instâncias 2 e 3 possuem uma simularidade: quase a totalidade dos acessos se encontram à distância 1. Isso indica que a localidade espacial deles é melhor que a das outras instâncias, por não terem que caminhar tanto na memória para acessar o próximo valor.

\begin{figure}
\centering
\mbox{\subfigure{\includegraphics[width=0.6\textwidth]{acessos1.png}}\quad
\subfigure{\includegraphics[width=0.6\textwidth]{acessos2.png} }}
\mbox{\subfigure{\includegraphics[width=0.6\textwidth]{acessos3.png} }\quad
\subfigure{\includegraphics[width=0.6\textwidth]{acessos4.png} }}
\caption{Distâncias de Acessos de todas as instâncias} \label{img_dist_acess}
\end{figure}


\subsection{Histogramas de Distância de Pilha}
\label{hist_dist_pilha}


Na Figura \ref{img_dist_pilha} foi agrupado os histogramas referentes ao calculo de distância de pilha de cada instância.

Nessa situação a instância que apresentou o melhor resultado foi a instância 2, pois a distância na pilha que mais correu foi o valor 3. Todas as outras distâncias ficaram bem distribuídas no histograma.

A segunda melhor instância nesse quesito seria a instância 4, pois muitos acessos se concentram em distâncias menores que 5. O resto dos acessos também se distribuíram bem pelo resto das distâncias.

As piores instâncias nessa análise foram as instâncias 1 e 3. A instãncia 1 possui muitos acessos de distâncias variadas, desde as menores até as maiores. Já a instância 3, apesar de possuir o ápice de acessos com distância de pilha igual a 1, pussui dezenas de acessos com distâncias maiores que 40, o que prejudica muito seu resultado final.

\begin{figure}
\centering
\mbox{\subfigure{\includegraphics[width=0.6\textwidth]{pilha1.png}}\quad
\subfigure{\includegraphics[width=0.6\textwidth]{pilha2.png} }}
\mbox{\subfigure{\includegraphics[width=0.6\textwidth]{pilha3.png} }\quad
\subfigure{\includegraphics[width=0.6\textwidth]{pilha4.png} }}
\caption{Distâncias de Pilha de todas as instâncias} \label{img_dist_pilha}
\end{figure}



\subsection{Resultado}

	Como é possível perceber pela série de testes na última seção, quanto maior a cadeia de caracteres a ser computada pelos algoritmos, mais homogênea e regular a diferença entre cada solução fica. Nos primeiros teste, de apenas uma palavra \ref{pequeno}, e de 5 palavras \ref{medio}, a variação é gritante. No primeiro varia de $0\%$ até $800\%$, no segundo a variação diminui mas ainda é relativamente grande. Podemos ver na Figura \ref{cincopalavras}, que a quantidade de caracteres vai de aproximadamente $15\%$ para $100\%$.

	Nos últimos testes a curva de variação de caracteres vai se tornando cada vez mais estável. No teste com dez palavras concatenadas, na Figura \ref{dezpalavras}, a variação vai de aproximadamente $25\%$ para $75\%$. Mas quando os caracteres vão para a casa das centenas e milhares, essa variação se torna cada vez menor, se localizando em torno dos $60\%$, $70\%$.


\section{CONCLUSÃO}
\label{conclusao}

    Os dois algoritmos implementados solucionam o problema do palíndromo apresentado. O primeiro algoritmo, utilizando o paradigma de programação dinâmica, possui solução ótima, ou seja, sempre encontrará o melhor resultado. Já o algoritmo guloso, possui solução não-ótima, pois toma as decisões que julga ser melhor naquele momento específico, porém sem considerar vários outros aspectos do problema.

    Foi visto na Avaliação Experimental, que quanto maior a cadeia de caracteres, mais constante ficava a diferença entre as soluções ótima e não-ótima. Claro que a solução ótima ainda conseguia resolver o problema com uma média de $60\%$ a menos de caracteres inseridos, mas ainda assim era um resultado que não era esperado. Acreditava que a solução não-ótima do algoritmo guloso teria um rendimento muito pior.

    Caso fosse acatar o resultado do teste \ref{pequeno}, poderia dizer que o teto observável para a quantidade de caracteres a ser inseridos seria $800\%$, porém esse foi um único caso dentre centenas de outros testados. Observando os resultados dos testes maiores, poderia dizer que o teto observável ficaria em torno de $75\%$.

    O problema de gerar palíndromos de forma eficiente foi solucionado com sucesso. Além disso, duas soluções, uma ótima e uma não-ótima, foram implementadas, obtendo um bom rendimento nos inúmeros testes realizados. Nos testes comparativos foi possível perceber a diferença gritante entre uma solução ótima e uma não-ótima.

\bibliographystyle{sbc}
\bibliography{tp3}

\end{document}
